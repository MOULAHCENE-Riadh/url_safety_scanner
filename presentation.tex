\documentclass{beamer}
\usetheme{metropolis} % or madrid

% Packages for professional look
\usepackage{graphicx}
\usepackage{booktabs}
\usepackage{lmodern}
\usepackage{hyperref}

% Title Information
\title{URL Safety Scanner}
\subtitle{A Project Presentation}
\author{Your Name}
\institute{Your Institution}
\date{\today}

\begin{document}

% Title Slide
\begin{frame}
  \titlepage
\end{frame}

% Table of Contents
\begin{frame}{Outline}
  \tableofcontents
\end{frame}

% Introduction
\section{Introduction}
\begin{frame}{Introduction}
  \begin{itemize}
    \item The internet is rife with malicious URLs posing threats to users.
    \item Manual verification of URL safety is impractical at scale.
    \item \textbf{URL Safety Scanner} aims to automate the detection of unsafe URLs using modern techniques.
    \item Motivation: Enhance user security and trust in web navigation.
  \end{itemize}
\end{frame}

% Problem Statement
\section{Problem Statement}
\begin{frame}{Problem Statement}
  \begin{itemize}
    \item \textbf{Challenge:} Identifying and blocking malicious URLs in real-time.
    \item Existing solutions may be slow, inaccurate, or require constant updates.
    \item Need for a scalable, accurate, and automated approach to URL safety assessment.
    \item Goal: Develop a system that efficiently classifies URLs as safe or unsafe.
  \end{itemize}
\end{frame}

% Methodology
\section{Methodology}
\begin{frame}{Methodology}
  \begin{itemize}
    \item \textbf{Data Collection:} Gathered datasets of labeled URLs (benign and malicious).
    \item \textbf{Feature Extraction:} Analyzed URL structure, lexical features, and metadata.
    \item \textbf{Machine Learning:} Trained classification models (e.g., Random Forest, SVM).
    \item \textbf{Evaluation:} Used metrics such as accuracy, precision, recall, and F1-score.
    \item \textbf{Tools:} Python, Scikit-learn, Pandas, and Matplotlib.
  \end{itemize}
\end{frame}

% Add: Feature Importance slide
\begin{frame}{Feature Importance}
    \begin{itemize}
        \item The model evaluates the importance of each extracted feature.
        \item Visualization helps understand which features contribute most to predictions.
    \end{itemize}
    \begin{figure}[h]
        \centering
        \includegraphics[width=0.8\textwidth]{figures/feature_importance.png} % Placeholder
        \caption{Feature importance as determined by the model.}
    \end{figure}
\end{frame}

% Add: Training and Validation Accuracy slide
\begin{frame}{Training and Validation Accuracy}
    \begin{itemize}
        \item Model performance is evaluated using training and validation accuracy curves.
        \item These curves help detect overfitting or underfitting.
    \end{itemize}
    \begin{figure}[h]
        \centering
        \includegraphics[width=0.8\textwidth]{figures/training_validation_accuracy.png} % Placeholder
        \caption{Training and validation accuracy over epochs.}
    \end{figure}
\end{frame}

% Implementation Details
\section{Implementation Details}
\begin{frame}{Implementation Details}
  \begin{itemize}
    \item \textbf{URL Scanning Pipeline:}
      \begin{itemize}
        \item Input: User-submitted URL.
        \item Preprocessing: Normalization and feature extraction.
        \item Classification: ML model predicts safety.
        \item Output: Safety verdict and explanation.
      \end{itemize}
    \item \textbf{Key Components:}
      \begin{itemize}
        \item Feature engineering (length, tokens, special characters, etc.)
        \item Model training and validation
        \item User interface for result display
      \end{itemize}
    \item \textbf{Extensibility:} Modular design for easy integration of new models or features.
  \end{itemize}
\end{frame}

% In Implementation Details, add a placeholder for the architecture/flowchart
\begin{frame}{System Architecture}
    \begin{itemize}
        \item Overview of the URL scanning pipeline.
        \item Shows data flow from input to prediction.
    \end{itemize}
    \begin{figure}[h]
        \centering
        \includegraphics[width=0.8\textwidth]{figures/system_architecture.png} % Placeholder
        \caption{System architecture and data flow.}
    \end{figure}
\end{frame}

% Results
\section{Results}
\begin{frame}{Results}
  \begin{itemize}
    \item Achieved high accuracy in classifying URLs.
    \item Robust performance across diverse datasets.
    \item Example metrics:
      \begin{itemize}
        \item Accuracy: 96\%
        \item Precision: 95\%
        \item Recall: 94\%
      \end{itemize}
    \item \textbf{Visualization:}
  \end{itemize}
  \begin{center}
    \includegraphics[width=0.7\linewidth]{results_chart_placeholder.png}
    % Replace with actual chart image
  \end{center}
\end{frame}

% In Results, add more graphics placeholders
\begin{frame}{Results: ROC Curve}
    \begin{itemize}
        \item Receiver Operating Characteristic (ROC) curve illustrates model performance.
    \end{itemize}
    \begin{figure}[h]
        \centering
        \includegraphics[width=0.8\textwidth]{figures/roc_curve.png} % Placeholder
        \caption{ROC curve for model performance.}
    \end{figure}
\end{frame}

\begin{frame}{Results: Confusion Matrix}
    \begin{itemize}
        \item Confusion matrix provides insight into classification errors.
    \end{itemize}
    \begin{figure}[h]
        \centering
        \includegraphics[width=0.8\textwidth]{figures/confusion_matrix.png} % Placeholder
        \caption{Confusion matrix of predictions.}
    \end{figure}
\end{frame}

% Conclusion
\section{Conclusion}
\begin{frame}{Conclusion}
  \begin{itemize}
    \item Developed an effective and scalable URL safety scanner.
    \item Demonstrated strong performance using machine learning techniques.
    \item \textbf{Future Work:}
      \begin{itemize}
        \item Integrate real-time threat intelligence feeds.
        \item Explore deep learning models for improved accuracy.
        \item Deploy as a browser extension or web service.
      \end{itemize}
    \item \textbf{Thank you!}
  \end{itemize}
\end{frame}

% Page numbers in footer
\addtobeamertemplate{footline}{%
  \begin{beamercolorbox}[wd=\paperwidth,ht=2.5ex,dp=1ex,leftskip=1em,rightskip=1em]{author in head/foot}
    \hfill\insertframenumber{} / \inserttotalframenumber
  \end{beamercolorbox}
}{}

\end{document}
